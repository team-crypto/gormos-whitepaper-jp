\section{ガバナンスとトークンユーティリティ}
ステークベースの投票を通じてオンチェーンガバナンスの対象となるいくつかのパラメータがあります。これらの投票シナリオのそれぞれは、ルートチェーンまたは\codenameのいずれかでスマートコントラクトを実装する一連の機能です。

\begin{itemize}
\item トークンペアの追加/削除: 新しいペアを追加するには、新しいシャードを作成し、バリデータプールを増やすか、シャードごとのバリデータの数を一時的に減らします。同様に、特定のシャードが非アクティブであるか、十分なアクティビティがない場合、Plasmaチェーンのステークホルダーで投票を行い、そのシャードをドロップするか他のシャードに統合します。

\item バリデータの登録の閾値: この値は、コミュニティの信頼度に基づき柔軟に設定されるべきであり、保証金やバリデータの扱うトランザクション量などを比較する必要があります。(デフォルトでは、バリデータになるためには最小限の障壁が指定されているべきであり、ガバナンス上の決定により閾値を上げていくのが望ましいでしょう)

\item バリデータプール: 各シャードに選ばれるバリデータの数を変更します。

\end{itemize}
\textbf{トークンユーティリティ.} KNCまたは\codenameの根底にあるトークンは、以下のものをはじめ、いくつかのユーティリティーに使用することができます。
\begin{itemize}
\item \codenameバリデータになるためのステーキング. これは、ルートチェーン(すなわちEthereum)上における\codenameのメインのコントラクトにKNCをデポジットすることでバリデータになれるという、基本的なステーキング機能です。
\item KNCを使って取引手数料を払い割引を受ける. このユーティリティは、Binance、Huobiなどのいくつかの一般的な取引所で使用されています。 これは\codenameで簡単に実装できます。
\end{itemize}
